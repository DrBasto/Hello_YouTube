\documentclass{article}
\usepackage[utf8]{inputenc}
\usepackage{amsmath}
\usepackage{graphicx}

\title{Hello YouTube}
\author{Sebastian Damian Romero Chavero}
\date{December 2022}

\begin{document}

\maketitle

% We can also put a star between section and the curly brackets to remove the numbering
\section{Definitions of $e$}
Let's begin with a formula $e^{i\pi}+1=0$. but we cam also do 

% We could also have used enumerate to have numbers instead of dots
\begin{enumerate}
    \item As a \textbf{limit}
    $$ e=\lim_{n\to\infty} \left(1+\frac{1}{n}\right)^n = 
    \lim_{n\to\infty}\frac{n}{\sqrt[n]{n!}}$$


    \item As a \textit{sum}:

    $$ e=\sum_{n=0}^{\infty} \frac{1}{n!} $$
    
    \item As a \underline{continued fraction}:

    $$e=2\frac{1}{1+\frac{1}{2+\frac{2}{3+\frac{3}{4+\frac{4}{5+\ddots}}}}}$$

\end{enumerate}

\section{More formulas}

$$\int_a^bf(x)dx$$
$$\iiint f(x,y,z)dxdydz$$
$$\vec{v}=<v_1, v_2, v_3>$$

$$vec{v}\cdot \Vec{w}$$

$$\begin{bmatrix}
1 & 2 & 3 \\
4 & 5 & 6 \\
\end{bmatrix}
$$

\section{Graphics}
\includegraphics[scale=0.30]{rwu_logo_hor-lila-cyan_rgb_0-2950056503.png}

\section{More on mathematics}
\begin{equation}
\label{limit}
e= \lim_{n\to\infty} \left(1+\frac{1}{n}\right)^n 
=\lim_{t\to 0}(1+t)^{\frac{1}{t}}
\end{equation}

Equation \ref{limit} was really cool!
\\2 and 3 are using align but are not group together
\begin{align}
e&= \lim_{n\to\infty} \left(1+\frac{1}{n}\right)^n\\
\text{Like and subscribe}&=\lim_{t\to 0}(1+t)^{\frac{1}{t}}
\end{align}
\\Using equation and then a split inside of it, we can group them together so it is easier to label them
\begin{equation}
\label{limit2}
\begin{split}
e&= \lim_{n\to\infty} \left(1+\frac{1}{n}\right)^n\\
\text{Like and subscribe}&=\lim_{t\to 0}(1+t)^{\frac{1}{t}}
\end{split}
\end{equation}

Equation \ref{limit2} is even cooler!



\end{document}
